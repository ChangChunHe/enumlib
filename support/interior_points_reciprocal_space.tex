\documentclass[11pt]{article}
\setlength{\topmargin}{-2cm}
\setlength{\oddsidemargin}{-1cm}
\setlength{\evensidemargin}{-1cm}
\setlength{\textwidth}{7.5in}
\setlength{\textheight}{10in}
\usepackage{amsmath,amssymb}
\pagestyle{empty}
\begin{document}
\noindent \textbf{Finding the $k$-grid points inside the reciprocal
  cell}\\

\vspace{0.1cm}
Let $L_K$ be the $k$-grid lattice. Let $\mathbb{K}$ be a matrix where the columns are the vectors of the
generating grid. The matrix $\mathbb{S}$ that transforms the $k$-grid
vectors into the reciprocal vectors must have all integer
elements. Its determinant is equal to the number of reducible
$k$-points.  Then, the recriprocal lattice, $L_R$, is given by
$\mathbb{R=KS}$ (where the columns of $\mathbb{R}$ are the reciprocal
lattice vectors).
 With no loss of generality, we may choose $\mathbb{S}$ as Hermite Normal
 Form matrix.

 
We want to know which points of $L_K$ are within the first unit cell of
$L_R$. Let a point within the cell be denoted $\vec x$. The (lattice) components
of $\vec x$ must be $0\leq x_i<1$. Since we are interested in points
in the cell that are also points of $L_K$, we have
$\mathbb{R}\vec x = \mathbb{K}\vec z$ where the components of $z$
are integers. (If so, then $\vec x$ is obviously a lattice point of the
$k$-grid [that is, $\vec x \in L_k$].)

But since $\mathbb{R=KS}$,\\
\parbox{2cm}{\ }
\parbox{5cm}{
\begin{eqnarray*}
\mathbb{R}\vec x& = & \mathbb{K}\vec z\\
\mathbb{KS}\vec x& = &\mathbb{K}\vec z\\
\mathbb{K}^{-1}\mathbb{KS}\vec x& = &\mathbb{K}^{-1}\mathbb{K}\vec z\\
\mathbb{S}\vec x& = &\vec z\\
\end{eqnarray*}
}$\Rightarrow$
\parbox{5cm}{\[
\left(\begin{array}{ccc}
a&0&0\\
b&c&0\\
d&e&f\\
\end{array}\right)
\left(\begin{array}{c}
x_1\\
x_2\\
x_3\\
\end{array}\right)
=
\left(\begin{array}{c}
z_1\\
z_2\\
z_3\\ 
\end{array}\right)\]}

So, 
\begin{eqnarray*}
ax_1&=&z_1\\
bx_1+cx_2&=&z_2\\
dx_1+ex_2+fx_3&=&z_3\\
\end{eqnarray*}

Since the components of $\vec x$ must be $[0,1)$,
\begin{eqnarray*}
0\leq x_1=z_1/a <
1&\rightarrow& \boxed{0\leq z_1<a}\\
0\leq x_2=\frac{z_2}{c}-\frac{b}{ca}z_1<1&\rightarrow&
\frac{b}{ca}z_1\leq \frac{z_2}{c} < 1+\frac{b}{ca}z_1\\ &\rightarrow&
\boxed{\frac{b}{a}z_1\leq z_2 < c+\frac{b}{a}z_1} \\
%\end{eqnarray*}
%\begin{eqnarray*}
0\leq x_3 =
\frac{z_3}{f}-\frac{d}{f}\frac{z_1}{a}-e[\frac{z_2}{c}-\frac{b}{ca}z_1]<1&&\\
\frac{d}{f}\frac{z_1}{a}+\frac{e}{f}[\frac{z_2}{c}-\frac{b}{ca}z_1]\leq
\frac{z_3}{f}<
1+\frac{d}{f}\frac{z_1}{a}+\frac{e}{f}[\frac{z_2}{c}-\frac{b}{ca}z_1]&&\\
d\frac{z_1}{a}+\frac{e}{c}[z_2-\frac{b}{a}z_1]\leq
z_3<
f+z_1\frac{d}{a}+\frac{e}{c}[z_2-\frac{b}{a}z_1]&&\\
%
\boxed{z_1[\frac{d}{a}-\frac{eb}{ca}]+\frac{e}{c}z_2\leq
z_3<
f+z_1[\frac{d}{a}-\frac{eb}{ca}]+\frac{e}{c}z_2}&&\\
\end{eqnarray*}

\end{document}
